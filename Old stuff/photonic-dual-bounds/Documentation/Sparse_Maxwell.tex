%\documentclass[12pt,twocolumn]{article}
\documentclass[12pt]{article}
\usepackage[margin=1in]{geometry} 
\usepackage{amsmath,amsthm,amssymb}
\usepackage{physics}
\usepackage{graphicx}
\usepackage{xcolor}
\usepackage{float}
\usepackage{subcaption}
\usepackage{verbatim}

\newcommand\numthis{\stepcounter{equation}\tag{\theequation}}
\newcommand\A{\mathbb{A}}
\newcommand\G{\mathbb{G}}
\newcommand\T{\mathbb{T}}
\newcommand\V{\mathbb{V}}
\newcommand\W{\mathbb{W}}
\newcommand\I{\mathbb{I}}
\newcommand\D{\mathbb{D}}
\newcommand\U{\mathbb{U}}
\newcommand\Proj{\mathbb{P}}
\newcommand\M{\mathcal{M}}
\newcommand\eps{\epsilon}
\newcommand\om{\omega}
\newcommand\Asym{\text{Asym}}
\newcommand\Sym{\text{Sym}}

\begin{document}
\title{Sparse computational formulation of dual bounds based on Maxwell Operator}
\author{Pengning Chao}
\maketitle

\section{Original Formulation with Explicit Green's Function}
In the original formulation, the primal optimization is over polarization currents represented by the image of the source under the $\T$ operator $\ket{\vb{T}}=\T \ket{\vb{S}}$. The fundamental scattering relation is
\begin{equation}
\I_d = \I_d(\V^{-1} - \G_{dd}) \T
\end{equation}
where $\I_d$ is the spatial projection onto the design region, $\V^{-1} = \chi^{-1} \I_d$, and $\G_{dd}$ is the Green's function restricted to the design domain. This can be generalized with the additional application an arbitrary operator $\Proj$ that commutes with $\I_d$:
\begin{equation}\label{eq:proj_scatter}
\Proj = \I_d (\Proj\V^{-1} - \Proj\G_{dd}) \T
\end{equation}
In practice $\Proj$ is often a spatial projection into a subregion of the entire design region. From this we can formulate scalar constraints of the form
\begin{equation}
\bra{\vb{S}}\Proj^\dagger\ket{\vb{T}} - \bra{\vb{T}}(\V^{-1}\Proj^\dagger - \G_{dd}\Proj^\dagger)\ket{\vb{T}} = 0
\end{equation}

\section{Sparse Formulation with Maxwell Operator}
The drawback to the original formulation is that $\G_{dd}$ is a dense matrix using a localize spatial basis representation, e.g., a finite difference grid. This leads poor scaling of dual optimization calculations with problem size. Noting that the inverse of the Green's function is proportional to the Maxwell operator $\M=(\curl\curl) - \eps_0 \om^2/c^2$, which is sparse under a localized spatial basis, we would like to reformulate the numerics based on $\M$. 

By pull out factors of $G_{dd}$ we can rewrite (\ref{eq:proj_scatter}) as
\begin{equation}
\G_{dd}^\dagger \G_{dd}^{\dagger-1} \Proj = \I_d \G_{dd}^\dagger (\G_{dd}^{\dagger-1} \Proj \V^{-1} \G_{dd}^{-1} - \G_{dd}^{\dagger-1}\Proj) \G_{dd} \T
\end{equation}
leading to scalar constraints of the form
\begin{equation}
\bra{\vb{S}} \Proj^\dagger \G_{dd}^{-1} (\G_{dd}\ket{\vb{T}}) - (\bra{\vb{T}}\G_{dd}^\dagger) (\G_{dd}^{\dagger-1} \V^{\dagger-1}\Proj^\dagger\G_{dd}^{-1} - \Proj^\dagger\G_{dd}^{-1}) (\G_{dd} \ket{\vb{T}})
\end{equation}
We can now declare that $\G_{dd} \ket{\vb{T}}$ will henceforth be our primal optimization variable. Now the dual optimization involves matrices composed of just $\V$, $\G_{dd}^{-1}$, and diagonal projections $\Proj$ which are all sparse, allowing for much better problem scaling.

\subsection{Computing $\G_{dd}^{-1}$}
From the basic relations $\M\vb{E} = i\om\vb{J}$ and $\vb{E} = (iZ/k)\G \vb{J}$ we have
\begin{equation}
\M\G = (k^2/\mu_0) \I
\end{equation}
where the un-subscripted operators are over all space. We divide space into the design region and background region, delimited by $d$ and $b$ subscripts, respectively. 

\begin{equation}
\G = \frac{k^2}{\mu_0} \M^{-1} = \frac{k^2}{\mu_0} \mqty[\M_{bb} & \M_{bd} \\ \M_{db} & \M_{dd}]^{-1}
\end{equation}
and making use of the block matrix inversion formula
\begin{equation}
\mqty[A & B \\ C & D]^{-1} = \mqty[A^{-1} + A^{-1}B(D-CA^{-1}B)^{-1}CA^{-1} & -A^{-1} B (D-CA^{-1}B)^{-1} \\ -(D-CA^{-1}B)^{-1}CA^{-1} & (D-CA^{-1}B)^{-1}]
\end{equation}
we have
\begin{align*}
G_{dd} &= \frac{k^2}{\mu_0} (\M_{dd} - \M_{db}\M_{bb}^{-1}\M_{bd})^{-1} \\
G_{dd}^{-1} &= \frac{\mu_0}{k^2} (\M_{dd} - \M_{db}\M_{bb}^{-1}\M_{bd}) \numthis
\end{align*}

In practice the background parts of $\M$ contain the boundary settings for the computational space, e.g., periodic boundary conditions or PML. For a spatially localized representation both $\M_{dd}$ and $\M_{db}\M_{bb}^{-1}\M_{bd}$ are sparse.
\end{document}